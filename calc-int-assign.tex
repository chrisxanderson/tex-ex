% LaTeX Calculus Template - Christopher Anderson http://github.com/chrisxanderson/tex-ex/
\documentclass[12pt,a4paper,anypage]{report}

% Packages
\usepackage{amsmath}
\usepackage{amssymb}
\usepackage{fancyhdr}
\usepackage{hyperref}
\usepackage{calc}
\usepackage[textheight=25.5cm,left=2.2cm,right=2.2cm,top=2.5cm,footskip=1.5cm]{geometry}

% Enviornment
\hypersetup{colorlinks,citecolor=violet,linkcolor=red,urlcolor=blue}

\pagestyle{fancy}
\fancyhf{}
\fancyhead[l]{ASSIGNMENT\\CLASS}
\fancyhead[r]{NAME\\\today}
\lfoot{\href{http://localhost/calculus.tex}{\LaTeX{}}
       \href{http://localhost/calculus.pdf}{PDF}
      }
\rfoot{\thepage}

\setlength{\parskip}{4.0mm}
\setlength{\parindent}{0.0mm}
\newcounter{questionnumber}
\setcounter{questionnumber}{0}
\newcounter{myenumeratecounter}
\setcounter{myenumeratecounter}{0}
\renewcommand{\themyenumeratecounter}{\alph{myenumeratecounter}}
\newcommand{\myitem}{\addtocounter{myenumeratecounter}{1} \themyenumeratecounter). \hspace{2mm}}
\newenvironment{question}[1]{\addtocounter{questionnumber}{1}{{\bf\thequestionnumber}) {\em{#1}}\setcounter{myenumeratecounter}{0}}}{}
\newcommand{\dd}{\displaystyle}

% Document
\begin{document}
\begin{question}{}
Let \boxed{u = \frac{\pi}{x}\text{ , } du=-\frac{\pi}{x^2}dx\text{ , and } -\frac{du}{\pi}=\frac{dx}{x^2}}

\begin{eqnarray*}
  \int{\frac{\cos\left(\frac{\pi}{x}\right)}{x^2}dx} &=& -\frac{1}{\pi}\int{\cos(u)du}\\
                                                     &=&-\frac{1}{\pi}\sin\left(\frac{\pi}{x}\right)+C\\
                                                     &=&-\frac{\sin\left(\frac{\pi}{x}\right)}{\pi}+C
\end{eqnarray*}

\end{question}

\begin{question}{}

\end{question}

% From Dr. Chris Hughes http://spot.pcc.edu/~chughes/latexdemofiles/252,PCC,integration,tips.tex - check his great LaTeX tutorial.
{\bf Integral Calculus Examples You May Need:}

`The Power Rule'
\[
\frac{d}{dx}x^n = nx^{n-1},
\]
where $n$ is any real number. Reversing the power rule to find an antiderivative is equally straight forward, and we have that
\[
\int x^n dx= \frac{x^{n+1}}{n+1} + C,
\]

Integration by Parts with the following choices of $u$ and $v$
\begin{eqnarray*}
  u = 2x, &&dv=e^x,\\
  du=2,&&v=e^x.
\end{eqnarray*}
This now gives
\begin{eqnarray*}
  \int x^2e^xdx&=&x^2e^x - \left(2xe^x - \int 2e^xdx \right)\\
  &=&x^2e^x-2xe^x+2e^x + C\\
  &=&(x^2-2x+2)e^x+C.
\end{eqnarray*}

Evaluating definite integrals using integration by parts is done in the following way. We consider
\[
\int_0^\pi x\sin(x)dx.
\]
Then from our work earlier on this handout, we have that
\begin{eqnarray*}
  \int_0^\pi x\sin(x)dx &=&\left[-x\cos(x)\right]^\pi_0 + \int_0^\pi \cos(x)dx \\
  &=&-\pi\cos(\pi) - ( -0\cos(0)) + [\sin(x)]_0^\pi\\
  &=&-\pi(-1) - 0 + (0-0)\\
  &=&\pi
\end{eqnarray*}

{\bf Trigonometric functions}

The following three identities (among others) are often useful when dealing with trigonometric functions
\begin{equation*}
  \left\{
  \begin{array}{rcl}
    \sin^2(\theta)+\cos^2(\theta)&=&1,\\
    \sin(m\theta \pm n\theta)&=&\sin(m\theta)\cos(n\theta)\pm\sin(n\theta)\cos (m\theta),\\
    \cos(m\theta\pm n\theta)&=&\cos (m\theta)\cos (n\theta) \mp \sin(m\theta)\sin (n\theta).
    \end{array}
  \right.
\end{equation*}

\end{document}



















